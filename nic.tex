\documentclass[11pt,a4j]{jarticle}
\title{ダニエル型電池と電気分解 \\ ~マイクロスケール実験を用いて考察する~} \author{2423 \ 張ニコラス謙豪 \\ 班員:土田\,雄輝 \ 鈴木\,陸馬 \ 鈴木\,勇哉} \date{実験日:2014/12/02 \\ 所要時間:1時間 \ 天候:晴れ \ 気温:18°C}
\begin{document}
\maketitle
\section{目的}
\begin{itemize}
\item ダニエル型電池の仕組みに関して理解を深める \item 実験での結果から値や式を求める
\item 化学反応によって起きた変化や結果をまとめる \end{itemize}
\section{実験概要}
現在、使用されている電池は、本格的な IT 社会を迎えるにあたって欠くことができない ハードの一つである。最新の実用電池としては、燃料電池を筆頭にリチウムイオン電池、 プラスチック電池等があり、その開発、改良が激しく行われている。今回の学生実験では、 マイクロスケール実験を用いて、ダニエル型電池の原理についての理解を深める。さらに 古楽的電池の環境負荷についても考察を行うとともに、電気分解の実験を行い、電気分解 に関しても考察する。
\newpage
\section{準備}
\begin{table}[htb]
\begin{center}
\caption{使用器具}
\begin{tabular}{|r|l|c|} \hline
\ & \ \ \ \ \ \ \ \ \ \ \ \ \ 品名 & 個数 \\ \hline 1 & USB クリップケーブル & 1 \\ \hline
2 & 試薬びん & 5 \\ \hline
3 & マイクロスケール用セル & 10 \\ \hline
4 & セル用ベース & 5 \\ \hline
5 & 炭素電極用芯 & 1セット \\ \hline
6 & 12セルプレート & 2 \\ \hline
7 & クリップケーブル & 1組 \\ \hline
8 & プロペラモーター & 1 \\ \hline
9 & 鉄板 & 5 \\ \hline
10 & 亜鉛板 & 5 \\ \hline
7
11 & ビスキングチューブ & 30cm \\ \hline 12 & ポリスポイト & 5 \\ \hline
13 & ミニピンセット & 5 \\ \hline \end{tabular}
\end{center}
\end{table}
\section{実験方法}
\subsection{ダニエル電池の作製}
○事前準備
\begin{itemize}
\item 電極の準備 \\
銅板、亜鉛板を準備する。\\ 電池の実験は、マイクロスケール用セル内で行う。準備した電極板は、マイクロスケール 用セルに、取り付けられるように加工しておくこと。加工は、電極板を端から 18mm のと ころで90度折り曲げる。電極板は薄いのでテで簡単に曲げることができる。 \\ 電極板は、マイクロスケール用セルに取り付けることが可能。電極板を差し込み蓋で固定 する。また、マイクロスケール用にセルに固定した電極板は、外に出ている部分が多少邪 魔なため、折り曲げておくと安定する。
\newpage
\item ダニエル電池電極の準備 \\
付属のビスキングチューブを用意する。25 から 30mm 程度をはさみで切ります。ビスキン グチューブは、チューブ状になっている。しかし、通常は乾燥した状態で張り付いてしまっ ているので、しばらく水につけて軽く揉む。するとチューブの状態になる。チューブ状のビ スキングチューブは、亜鉛電極にかぶらない用に差し込むこと。その後電池の状態にセッ ト押す。\\ ※ビスキングチューブは、半透明の性質をもっており、溶液が混ざるのを防ぎ、イオンを 通す仕組みになっています。 \\
\item 試薬の準備 \\
ダニエル電池で使用する試薬を調整して準備する。 \\
\ 硫酸銅水溶液:1.0mol/l \\
\ 硫酸亜鉛水溶液:0.1mol/l
\item 使用する実験器具 \\ マイクロスケール用セル、鉄板、亜鉛板、ビスキングチューブ、セル用ベース、プロペラ モーター、ポリスポイト、ミニピンセット、クリップケーブル
\end{itemize}
○実験方法 \par
8

\ \ 1,マイクロスケール要セルに差し込んだ銅板とビスキングチューブ付きの亜鉛板を 確認する。\par
\ \ 2,準備した溶液をビスキングチューブ内と外にそれぞれにポリスポイトで1mlずつ 入れる。まず銅板は一度外すように。ビスキングチューブ外側には、硫酸銅水溶液を入れ る。亜鉛板 (ビスキングチューブ内) には、硫酸亜鉛水溶液を入れる。なお、亜鉛板 (ビス キングチューブ内) にいれるときは、ピンセットで軽くつまんでからポリスポイトに差し 込んで溶液を入れる。\par
\ \ 3,溶液をセットしたら蓋でそれぞれの金属板を固定すればダニエル電池の完成。\par
\ \ 4,電極にプロペラモーターを接続して、プロペラが回るかどうか確認するように。 デジタルマルチメータで電圧を測定すると、状態を更に確認できます。亜鉛板・銅板のど ちらが正極か負極なのかを考えながら接続するように。
\newpage \subsection{電気分解の実験} ○事前準備
\begin{itemize}
\item 炭素電極の準備 \\
実験 1 で使用した炭素電極を使用する。
\item 試薬の準備 \\ 電気分解の実験では、代表的な塩化銅水溶液の電気分解を行う。\par \ \ 塩化銅水溶液:0.5mol/l
\item 実験容器の準備 \par
\ \ 1,蓋を外し炭素電極を差し込んだ蓋をマイクロスケール用セルに差し込む。\par \ \ 2,この状態では不安定なので、セル用ベースの穴の部分にさしこんでください。
\item 使用する実験器具 \\ マイクロスケール用セル、セル用ベース、炭素電極用芯、USB クリップケーブル、クリップ ケーブル、ミニピンセット、ポリスポイト、ろ紙、赤色色素に浸けたろ紙、ストップウォッ チ、USBハブ \\ \\
\end{itemize}
○実験方法 \par
\ \ 1,マイクロスケール用セルの中に、塩化銅水溶液2ml程度入れ、用意した炭素電極 をしっかり差し込む。また、電極をつけていないセルの中にも同じ量の塩化銅水溶液を入 れて、比較用のセルも作っておく。\par
\ \ 2,USB電源に差し込んだUSBクリップケーブルの赤と黒のクリップを炭素電極の両端 に接続する。\par
\ \ 3,USB電源のスイッチをいれて、2分間電気分解を行う。\par
\ \ 4,電気分解が終わったらスイッチをOFFにしてから電極からクリップをはずし、炭
9

素電極を蓋ごとはずしてろ紙の上に置く。\par
\ \ 5,赤色色素液に浸けたろ紙をピンセットで取り出し、セルの蓋のするようにかぶす。 \par
\ \ 6, 炭素電極に析出した物質を調べます。爪楊枝などを使って電極をこすり、析出し た物質をろ紙の上に集める。集めたものをガラス試験官の底でこすって変化を確認する。 \par
\ \ 7,マイクロスケール用セル内に残った水溶液の色は、電気分解の前後でどのように 変化したかを比べましょう。その際、1 で準備した比較用のセルと並べて比較すればわか りやすくあります。
\newpage \section{実験結果} ★ダニエル型電池\par
\ \ (1),それぞれの電極ではどのような変化が起きたか。\par \ \ \ \ 鉄板側:変化なし \\ \par
\ \ \ \ 亜鉛板側:黒ずんだ \\ \par
\ \ (2),今回作成したマイクロスケールのダニエル型電池の電圧(起電力)と電流を測 定せよ。 \par
\ \ \ \ 電圧:1.0V \ , \ 電流:9mA \\ \par
\ \ (3),ボルタ電池は、$(-)Zn|H_2SO_4(aq)|Cu(+)$と表すことができる。\\
\ \ \ \ \ これにならって、今回製作したダニエル電池の電池式を以下に書け。 \par
\ \ \ \ \ \ \ \ \ \ \ \ \ \ \ \ \ \ \ $(-)Zn|ZnSO_4CuSO_4(aq)|Cu(+)$ \\ \par
\ \ (4),ボルタ電池の負極、正極におけるイオン式(酸化還元式)と全体の式は、いか に示す通りである。 \par
\ \ \ \ $負極(Zn):Zn \ → \ Zn^{2+} + 2e^- \ (酸化) \ \ \ \ 単極電位:E_{zn}=(-0.763)V$ \p \ \ \ \ $正極(Cu):2H^+ \ + 2e^- \ → \ H_2 \ (還元) \ \ \ \ 単極電位:E_{cu}=(+0.340)V$ \pa \ \ \ \ $全体の反応:Zn \ + 2H^+ \ → \ Zn^{2+} \ + \ H_2 \ \ \ ∴起電力 E=E_{cu}-E_{zn}=(+1.
★電気分解\par
\ \ (1),電気分解の結果、それぞれの極ではどのような変化が起こったか。両極の様子
を書け\par
\ \ \ \ 陽極:泡が電源を入れてすぐに出るようになった。、炭素棒にザラザラしてい
るような物質がついてるみたいに見えた。\\ \par
\ \ \ \ 陰極:赤色の物質が変化する。褐色物質が析出できた。\\ \par
\ \ (2),この電気分解で起きてる陰極と陽極のイオン反応式を書け。\par \ \ \ \ $陽極:2Cl^- \ → \ Cl_2 \ + \ 2e^-$
\ \ \ \ $陰極:Cu^{2+} \ + \ 2e^- \ → \ Cu$ \\ \par
10

\ \ (3),赤色の色素のついたろ紙はどうなったかを書け \par \ \ \ \ 変化なし \\ \par
\ \ (4),ろ紙上で電極に接既出した物質をガラス製試験管の底でこすった結果、どのよ うな変化が見られたか \par
\ \ \ \ 金属光沢が出た \\ \par
\ \ (5),水溶液の色は電気分解の前後でどのように変わったかを書け。\par
\ \ \ \ ほとんど変化なし。
\newpage
\section{考察} どちらの実験もうまくいき、他の班と比べても値や変化、結果などもあまり変わりはな かった
\section{あとがき} 反省点:溶液の入れる量が少し足りなかった \par 感想と意見:目の前で科学実験による電気発生がみれてよかった
\end{document}
